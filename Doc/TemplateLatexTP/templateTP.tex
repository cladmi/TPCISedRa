%%%%%%%%%%%%%%%%%%%%%%%%%%%%%%%%%%%%%%%%%
% Programming/Coding Assignment
% LaTeX Template
%
% This template has been downloaded from:
% http://www.latextemplates.com
%
% Original author:
% Ted Pavlic (http://www.tedpavlic.com)
%
% Note:
% The \lipsum[#] commands throughout this template generate dummy text
% to fill the template out. These commands should all be removed when 
% writing assignment content.
%
% This template uses a Perl script as an example snippet of code, most other
% languages are also usable. Configure them in the "CODE INCLUSION 
% CONFIGURATION" section.
%
%%%%%%%%%%%%%%%%%%%%%%%%%%%%%%%%%%%%%%%%%

%----------------------------------------------------------------------------------------
%	PACKAGES AND OTHER DOCUMENT CONFIGURATIONS
%----------------------------------------------------------------------------------------

\documentclass{article}

\usepackage{fancyhdr} % Required for custom headers
\usepackage{lastpage} % Required to determine the last page for the footer
\usepackage{extramarks} % Required for headers and footers
\usepackage[usenames,dvipsnames]{color} % Required for custom colors
\usepackage{graphicx} % Required to insert images
\usepackage{listings} % Required for insertion of code
\usepackage{courier} % Required for the courier font
\usepackage{lipsum} % Used for inserting dummy 'Lorem ipsum' text into the template

% Margins
\topmargin=-0.45in
\evensidemargin=0in
\oddsidemargin=0in
\textwidth=6.5in
\textheight=9.0in
\headsep=0.25in

\linespread{1.1} % Line spacing

% Set up the header and footer
\pagestyle{fancy}
%\lhead{\hmwkAuthorName} % Top left header
%\chead{\hmwkClass\ (\hmwkClassInstructor\ \hmwkClassTime): \hmwkTitle} % Top center head
\lhead{\hmwkTitle} % Top center head
\rhead{\firstxmark} % Top right header
\lfoot{\lastxmark} % Bottom left footer
\cfoot{} % Bottom center footer
\rfoot{Page\ \thepage\ of\ \protect\pageref{LastPage}} % Bottom right footer
\renewcommand\headrulewidth{0.4pt} % Size of the header rule
\renewcommand\footrulewidth{0.4pt} % Size of the footer rule

\setlength\parindent{0pt} % Removes all indentation from paragraphs

%----------------------------------------------------------------------------------------
%	CODE INCLUSION CONFIGURATION
%----------------------------------------------------------------------------------------

\definecolor{MyDarkGreen}{rgb}{0.0,0.4,0.0} % This is the color used for comments
\lstloadlanguages{Perl, Java, Python, C++} % Load Perl syntax for listings, for a list of other languages supported see: ftp://ftp.tex.ac.uk/tex-archive/macros/latex/contrib/listings/listings.pdf

\lstset{language=C++, % Use C++
        frame=single, % Single frame around code
        basicstyle=\small\ttfamily, % Use small true type font
        keywordstyle=[1]\color{Blue}\bf, % functions bold and blue
        keywordstyle=[2]\color{Purple}, % function arguments purple
        keywordstyle=[3]\color{Blue}\underbar, % Custom functions underlined and blue
        identifierstyle=, % Nothing special about identifiers                                         
        commentstyle=\usefont{T1}{pcr}{m}{sl}\color{MyDarkGreen}\small, % Comments small dark green courier font
        stringstyle=\color{Purple}, % Strings are purple
        showstringspaces=false, % Don't put marks in string spaces
        tabsize=4, % 5 spaces per tab
       	%
        morecomment=[l][\color{Blue}]{...}, % Line continuation (...) like blue comment
        numbers=left, % Line numbers on left
        firstnumber=1, % Line numbers start with line 1
        numberstyle=\tiny\color{Blue}, % Line numbers are blue and small
        stepnumber=2 % Line numbers go in steps of 5
}
\lstset{language=Java, % Use Java in this example
        frame=single, % Single frame around code
        basicstyle=\small\ttfamily, % Use small true type font
        keywordstyle=[1]\color{Blue}\bf, % functions bold and blue
        keywordstyle=[2]\color{Purple}, %  function arguments purple
        keywordstyle=[3]\color{Blue}\underbar, % Custom functions underlined and blue
        identifierstyle=, % Nothing special about identifiers                                         
        commentstyle=\usefont{T1}{pcr}{m}{sl}\color{MyDarkGreen}\small, % Comments small dark green courier font
        stringstyle=\color{Purple}, % Strings are purple
        showstringspaces=false, % Don't put marks in string spaces
        tabsize=4, % 5 spaces per tab
       	%
        morecomment=[l][\color{Blue}]{...}, % Line continuation (...) like blue comment
        numbers=left, % Line numbers on left
        firstnumber=1, % Line numbers start with line 1
        numberstyle=\tiny\color{Blue}, % Line numbers are blue and small
        stepnumber=2 % Line numbers go in steps of 5
}
\lstset{language=Python, % Use Python in this example
        frame=single, % Single frame around code
        basicstyle=\small\ttfamily, % Use small true type font
        keywordstyle=[1]\color{Blue}\bf, % functions bold and blue
        keywordstyle=[2]\color{Purple}, % function arguments purple
        keywordstyle=[3]\color{Blue}\underbar, % Custom functions underlined and blue
        identifierstyle=, % Nothing special about identifiers                                         
        commentstyle=\usefont{T1}{pcr}{m}{sl}\color{MyDarkGreen}\small, % Comments small dark green courier font
        stringstyle=\color{Purple}, % Strings are purple
        showstringspaces=false, % Don't put marks in string spaces
        tabsize=4, % 5 spaces per tab
       	%
        morecomment=[l][\color{Blue}]{...}, % Line continuation (...) like blue comment
        numbers=left, % Line numbers on left
        firstnumber=1, % Line numbers start with line 1
        numberstyle=\tiny\color{Blue}, % Line numbers are blue and small
        stepnumber=2 % Line numbers go in steps of 2
}

\lstset{language=Perl, % Use Perl in this example
        frame=single, % Single frame around code
        basicstyle=\small\ttfamily, % Use small true type font
        keywordstyle=[1]\color{Blue}\bf, % Perl functions bold and blue
        keywordstyle=[2]\color{Purple}, % Perl function arguments purple
        keywordstyle=[3]\color{Blue}\underbar, % Custom functions underlined and blue
        identifierstyle=, % Nothing special about identifiers                                         
        commentstyle=\usefont{T1}{pcr}{m}{sl}\color{MyDarkGreen}\small, % Comments small dark green courier font
        stringstyle=\color{Purple}, % Strings are purple
        showstringspaces=false, % Don't put marks in string spaces
        tabsize=5, % 5 spaces per tab
        %
        % Put standard Perl functions not included in the default language here
        morekeywords={rand},
        %
        % Put Perl function parameters here
        morekeywords=[2]{on, off, interp},
        %
        % Put user defined functions here
        morekeywords=[3]{test},
       	%
        morecomment=[l][\color{Blue}]{...}, % Line continuation (...) like blue comment
        numbers=left, % Line numbers on left
        firstnumber=1, % Line numbers start with line 1
        numberstyle=\tiny\color{Blue}, % Line numbers are blue and small
        stepnumber=5 % Line numbers go in steps of 5
}

% Creates a new command to include a perl script, the first parameter is the filename of the script (without .pl), the second parameter is the caption
\newcommand{\perlscript}[2]{
\begin{itemize}
\item[]\lstinputlisting[caption=#2,label=#1]{#1.pl}
\end{itemize}
}

\newcommand{\javascript}[2]{
\begin{itemize}
\item[]\lstinputlisting[caption=#2,label=#1]{#1.java}
\end{itemize}
}

\newcommand{\pythonscript}[2]{
\begin{itemize}
\item[]\lstinputlisting[caption=#2,label=#1]{#1.py}
\end{itemize}
}

\newcommand{\cppscript}[2]{
\begin{itemize}
\item[]\lstinputlisting[caption=#2,label=#1]{#1.cpp}
\end{itemize}
}

%----------------------------------------------------------------------------------------
%	DOCUMENT STRUCTURE COMMANDS
%	Skip this unless you know what you're doing
%----------------------------------------------------------------------------------------

% Header and footer for when a page split occurs within a problem environment
\newcommand{\enterProblemHeader}[1]{
\nobreak\extramarks{#1}{#1 continued on next page\ldots}\nobreak
\nobreak\extramarks{#1 (continued)}{#1 continued on next page\ldots}\nobreak
}

% Header and footer for when a page split occurs between problem environments
\newcommand{\exitProblemHeader}[1]{
\nobreak\extramarks{#1 (continued)}{#1 continued on next page\ldots}\nobreak
\nobreak\extramarks{#1}{}\nobreak
}

\setcounter{secnumdepth}{0} % Removes default section numbers
\newcounter{homeworkProblemCounter} % Creates a counter to keep track of the number of problems

\newcommand{\homeworkProblemName}{}
\newenvironment{homeworkProblem}[1][Problem \arabic{homeworkProblemCounter}]{ % Makes a new environment called homeworkProblem which takes 1 argument (custom name) but the default is "Problem #"
\stepcounter{homeworkProblemCounter} % Increase counter for number of problems
\renewcommand{\homeworkProblemName}{#1} % Assign \homeworkProblemName the name of the problem
\section{\homeworkProblemName} % Make a section in the document with the custom problem count
\enterProblemHeader{\homeworkProblemName} % Header and footer within the environment
}{
\exitProblemHeader{\homeworkProblemName} % Header and footer after the environment
}

\newcommand{\problemAnswer}[1]{ % Defines the problem answer command with the content as the only argument
\noindent\framebox[\columnwidth][c]{\begin{minipage}{0.98\columnwidth}#1\end{minipage}} % Makes the box around the problem answer and puts the content inside
}

\newcommand{\homeworkSectionName}{}
\newenvironment{homeworkSection}[1]{ % New environment for sections within homework problems, takes 1 argument - the name of the section
\renewcommand{\homeworkSectionName}{#1} % Assign \homeworkSectionName to the name of the section from the environment argument
\subsection{\homeworkSectionName} % Make a subsection with the custom name of the subsection
\enterProblemHeader{\homeworkProblemName\ [\homeworkSectionName]} % Header and footer within the environment
}{
\enterProblemHeader{\homeworkProblemName} % Header and footer after the environment
}

%----------------------------------------------------------------------------------------
%	NAME AND CLASS SECTION
%----------------------------------------------------------------------------------------

\newcommand{\hmwkTitle}{Java/C++/Python Exercices} % Assignment title
\newcommand{\hmwkDueDate}{Thursday,\ October\ 29,\ 2015} % Due date
\newcommand{\hmwkClass}{Continuous Integration Seminar} % Course/class
\newcommand{\hmwkClassTime}{13h30} % Class/lecture time
\newcommand{\hmwkClassInstructor}{} % Teacher/lecturer
\newcommand{\hmwkAuthorName}{Maurice Bremond, Gaetan Harter, David Parsons -- SED Inria Rhone-Alpes} % Your name

%----------------------------------------------------------------------------------------
%	TITLE PAGE
%----------------------------------------------------------------------------------------

\title{
%\vspace{2in}
\textsc{\hmwkClass}\\
\textbf{\hmwkTitle}\\
%\normalsize\vspace{0.1in}\small{\hmwkDueDate}\\
%\vspace{0.1in}\large{\textit{\hmwkClassInstructor\ \hmwkClassTime}}
%\vspace{3in}
}

\author{\textsf{\hmwkAuthorName}}
\date{\textit{\hmwkDueDate}} % Insert date here if you want it to appear below your name

%----------------------------------------------------------------------------------------

\begin{document}

\maketitle

%----------------------------------------------------------------------------------------
%	TABLE OF CONTENTS
%----------------------------------------------------------------------------------------

%\setcounter{tocdepth}{1} % Uncomment this line if you don't want subsections listed in the ToC

%\newpage
\tableofcontents
%\newpage


%----------------------------------------------------------------------------------------
%	PROBLEM 1
%----------------------------------------------------------------------------------------

% To have just one problem per page, simply put a \clearpage after each problem
\begin{homeworkProblem}

\subsection{Setup}

\subsubsection{Clone the git repository from inria forge and create your own branch}
\begin{lstlisting}
git clone git+ssh://<yourloginhere>@scm.gforge.inria.fr/gitroot/tpcisedra/tpcisedra.git 
cd tpcisedra/java
git checkout -b <yournamehere>
\end{lstlisting}
%\vspace{1em}
The project is made up of:
\begin{itemize}
\item A Maven ``Project Object Model'' file pom.xml
\item A class Sphere and its test class SphereTest
\end{itemize}

\subsubsection{Check that you can build the project}
\begin{lstlisting}
mvn package
\end{lstlisting}

You should see these lines (among others):
\begin{lstlisting}
[INFO] ------------------------------------------------------------------------
[INFO] BUILD SUCCESS
[INFO] ------------------------------------------------------------------------
\end{lstlisting}



%--------------------------------------
\subsection{Testing your code}

\subsubsection{Run the (single) test}
\begin{lstlisting}
mvn test
\end{lstlisting}

You should see something like
\begin{lstlisting}
-------------------------------------------------------
 T E S T S
-------------------------------------------------------
Running fr.inria.sed.SphereTest
Tests run: 1, Failures: 0, Errors: 0, Skipped: 0, Time elapsed: 0.069 sec

Results :

Tests run: 1, Failures: 0, Errors: 0, Skipped: 0

[INFO] ------------------------------------------------------------------------
[INFO] BUILD SUCCESS
[INFO] ------------------------------------------------------------------------
\end{lstlisting}

\subsubsection{Have a look at the code of Sphere::computeVolume()}
\begin{lstlisting}
public double computeVolume1() {
  return 4 * Math.PI * Math.pow(radius_, 3) / 3;
}
\end{lstlisting}
We might want \verb?4 * Math.PI / 3? to be computed once and for all.

\subsubsection{Extract this value into a class data member and run the test again, it should fail:}
\begin{lstlisting}
-------------------------------------------------------
 T E S T S
-------------------------------------------------------
Running fr.inria.sed.SphereTest
Tests run: 1, Failures: 1, Errors: 0, Skipped: 0, Time elapsed: 0.076 sec <<< FAILURE!

Results :

Failed tests:   testComputeVolume(fr.inria.sed.SphereTest):
    expected:<14.137166941154069> but was:<14.137166941154067>

Tests run: 1, Failures: 1, Errors: 0, Skipped: 0

[INFO] ------------------------------------------------------------------------
[INFO] BUILD FAILURE
[INFO] ------------------------------------------------------------------------
\end{lstlisting}
This is because of floating point arithmetics \emph{rouding errors}.

Our response to this issue strongly depends on what kind of application we are working on.
In some cases, we want to be aware that something has changed, even when the change is the tiniest.
In that case, the test we already have is just what we want.

In other cases however, we need not worry about such a small difference and hence don't want to be bothered by our tests complaining.
Find a way to deal with the latter case.



%--------------------------------------
\subsection{Test Coverage}
Very well, all our tests pass, but what does that mean regarding our application ?
Not much you may say, since we only have one very simple test. But can you quantify it and how will you tell on a real project ?

This is when test coverage becomes handy.

\subsubsection{Run the following maven command:}
\begin{lstlisting}
mvn cobertura:cobertura
\end{lstlisting}
You can then visualize the results in an ad-hoc microsite
\begin{lstlisting}
iceweasel target/site/cobertura/index.html &
\end{lstlisting}

\subsubsection{Populate your tests to achieve 100\% test coverage.}

\subsubsection{[Optional] Integrate cobertura to your reporting local website}
Add the following to your pom.xml
\begin{lstlisting}
  <reporting>
    <plugins>
      <plugin>
        <groupId>org.codehaus.mojo</groupId>
        <artifactId>cobertura-maven-plugin</artifactId>
        <version>2.7</version>
      </plugin>
    </plugins>
  </reporting>
\end{lstlisting}
The report generation is now included in the build lifecycle (``site'' phase). To generate and visualize it:
\begin{lstlisting}
mvn site
iceweasel target/site/project-reports.html &
\end{lstlisting}


\subsubsection{Let's add a new class (Alphabet) and its test class AlphabetTest to our project}
\begin{lstlisting}
git merge alpha
\end{lstlisting}

\subsubsection{Generate and visualize the corresponding coverage report}

\subsubsection{Make what changes are necessary to achieve 100\% test coverage}
Hint: you may not need any additional tests ;)



%--------------------------------------
\subsection{Stylecheck}

\subsubsection{Let's try and run a style checker:}
\begin{lstlisting}
mvn checkstyle:checkstyle
\end{lstlisting}
We get plenty of errors with the default style \verb?sun_checks.xml?.

Our coding style is close to google style than it is to sun style.

\subsubsection{Try running with google checks (provided by the plugin)}
\begin{lstlisting}
mvn checkstyle:checkstyle -Dcheckstyle.config.location=google_checks.xml
\end{lstlisting}
NOTE: this can also be achieved by adding the following lines to your pom.xml:
\begin{lstlisting}
  <properties>
    <checkstyle.config.location>checkstyle-test.xml</checkstyle.config.location>
  </properties>
\end{lstlisting}
It's getting better but we might want to make a few changes to this default.

\subsubsection{Retrieve a local copy of google checks}
\begin{lstlisting}
wget https://raw.githubusercontent.com/checkstyle/checkstyle/
     18f6ebbcad23e88e3ae30fc79be464b8b7772a0d/google_checks.xml
mv google_checks.xml  accordingly
\end{lstlisting}

\subsubsection{Modify it so that it accepts single character parameter names}
You may also consider removing trailing underscores from data members or amending checkstyle.xml


\subsubsection{[Optional] Integrate checkstyle to your reporting local website}
Add the following to your pom.xml
\begin{lstlisting}
  <reporting>
    <plugins>
      <plugin>
        <groupId>org.apache.maven.plugins</groupId>
        <artifactId>maven-checkstyle-plugin</artifactId>
        <version>2.16</version>
      </plugin>
    </plugins>
  </reporting>
\end{lstlisting}
The report generation is now included in the build lifecycle (``site'' phase). To generate and visualize it:
\begin{lstlisting}
mvn site
iceweasel target/site/project-reports.html &
\end{lstlisting}
\end{homeworkProblem}

\clearpage




\begin{homeworkProblem}
Listing \ref{homework_example} shows a Perl script.

\perlscript{homework_example}{Sample Perl script with highlighting}

\lipsum[1]

Listing \ref{Sphere} shows a Java program.
\javascript{Sphere}{Sample Java program with Highlighting}

Another program example for Python
Listing \ref{Python_example} shows a Python program.
\pythonscript{Python_example}{Sample Python program with highlighting}


\end{homeworkProblem}

%----------------------------------------------------------------------------------------
%	PROBLEM 2
%----------------------------------------------------------------------------------------

\begin{homeworkProblem}
\lipsum[2]

Extract from C++ program
\begin{lstlisting}[name=Function Test, frame=trBL]
#include <iostream>
#include <basetsd.h>
#include <iomanip>
#include <cstdlib>
using namespace std;
 
union FloatNum //Here the tag name (FloatNum) is redundant.
 {
   float fx;//4 bytes variable
   long  lx;//4 bytes variable
 }fn;
 
union DoubleNum
 {
   double dx;  //8 bytes variable
   LONG64 lx;  //8 bytes variable
 }dn;
 
union LongDoubleNum
 {
   long double dx;  //12 bytes variable
   long  lx[3]; // 3 * 4 bytes variable
 }ldn;
 
int main()
{
    fn.fx = -118.6253433; //variable assignment declaration statement
    //show size of float
    cout << "\nsize of float = " << dec << sizeof(fn.fx) << endl;
    cout << setprecision(10) << fn.fx << " = 0x" << hex << fn.lx << endl;
 
    dn.dx =  112.6255678;  //assign value to a variable
    //show size of double
    cout << "\nsize of double = " << dec << sizeof(dn.dx) << endl;
    cout << dn.dx <<"  = 0x" << hex << dn.lx << endl;
 
    ldn.dx = -12.61256125;  //assign value to a variable
    //show size of long double
    cout << "\nsize of long double = " << dec << sizeof(ldn.dx) << endl;
    cout << setprecision(10) << ldn.dx << " = 0x" 
         <<  hex << ldn.lx[2] << ldn.lx[1] << ldn.lx[0] << endl;
    return 0;
}
\end{lstlisting}


\problemAnswer{

Answer
\begin{center}
\includegraphics[width=0.75\columnwidth]{example_figure} % Example image
\end{center}

\lipsum[3-5]
}
\end{homeworkProblem}

%----------------------------------------------------------------------------------------

\end{document}
