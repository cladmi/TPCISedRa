%%%%%%%%%%%%%%%%%%%%%%%%%%%%%%%%%%%%%%%%%
% Programming/Coding Assignment
% LaTeX Template
%
% This template has been downloaded from:
% http://www.latextemplates.com
%
% Original author:
% Ted Pavlic (http://www.tedpavlic.com)
%
% Note:
% The \lipsum[#] commands throughout this template generate dummy text
% to fill the template out. These commands should all be removed when 
% writing assignment content.
%
% This template uses a Perl script as an example snippet of code, most other
% languages are also usable. Configure them in the "CODE INCLUSION 
% CONFIGURATION" section.
%
%%%%%%%%%%%%%%%%%%%%%%%%%%%%%%%%%%%%%%%%%

%----------------------------------------------------------------------------------------
%	PACKAGES AND OTHER DOCUMENT CONFIGURATIONS
%----------------------------------------------------------------------------------------

\documentclass{article}

\usepackage{fancyhdr} % Required for custom headers
\usepackage{lastpage} % Required to determine the last page for the footer
\usepackage{extramarks} % Required for headers and footers
\usepackage[usenames,dvipsnames]{color} % Required for custom colors
\usepackage{graphicx} % Required to insert images
\usepackage{listings} % Required for insertion of code
\usepackage{courier} % Required for the courier font
\usepackage{lipsum} % Used for inserting dummy 'Lorem ipsum' text into the template
\usepackage{dirtree}

% Margins
\topmargin=-0.45in
\evensidemargin=0in
\oddsidemargin=0in
\textwidth=6.5in
\textheight=9.0in
\headsep=0.25in

\linespread{1.1} % Line spacing

% Set up the header and footer
\pagestyle{fancy}
%\lhead{\hmwkAuthorName} % Top left header
%\chead{\hmwkClass\ (\hmwkClassInstructor\ \hmwkClassTime): \hmwkTitle} % Top center head
\lhead{\hmwkTitle} % Top center head
\rhead{\firstxmark} % Top right header
\lfoot{\lastxmark} % Bottom left footer
\cfoot{} % Bottom center footer
\rfoot{Page\ \thepage\ of\ \protect\pageref{LastPage}} % Bottom right footer
\renewcommand\headrulewidth{0.4pt} % Size of the header rule
\renewcommand\footrulewidth{0.4pt} % Size of the footer rule

\setlength\parindent{0pt} % Removes all indentation from paragraphs

%----------------------------------------------------------------------------------------
%	CODE INCLUSION CONFIGURATION
%----------------------------------------------------------------------------------------

\definecolor{MyDarkGreen}{rgb}{0.0,0.4,0.0} % This is the color used for comments
\lstloadlanguages{Perl, Java, Python, C++} % Load Perl syntax for listings, for a list of other languages supported see: ftp://ftp.tex.ac.uk/tex-archive/macros/latex/contrib/listings/listings.pdf

\lstset{language=C++, % Use C++
        frame=single, % Single frame around code
        basicstyle=\small\ttfamily, % Use small true type font
        keywordstyle=[1]\color{Blue}\bf, % functions bold and blue
        keywordstyle=[2]\color{Purple}, % function arguments purple
        keywordstyle=[3]\color{Blue}\underbar, % Custom functions underlined and blue
        identifierstyle=, % Nothing special about identifiers                                         
        commentstyle=\usefont{T1}{pcr}{m}{sl}\color{MyDarkGreen}\small, % Comments small dark green courier font
        stringstyle=\color{Purple}, % Strings are purple
        showstringspaces=false, % Don't put marks in string spaces
        tabsize=4, % 5 spaces per tab
       	%
        morecomment=[l][\color{Blue}]{...}, % Line continuation (...) like blue comment
        numbers=left, % Line numbers on left
        firstnumber=1, % Line numbers start with line 1
        numberstyle=\tiny\color{Blue}, % Line numbers are blue and small
        stepnumber=2 % Line numbers go in steps of 5
}
\lstset{language=Java, % Use Java in this example
        frame=single, % Single frame around code
        basicstyle=\small\ttfamily, % Use small true type font
        keywordstyle=[1]\color{Blue}\bf, % functions bold and blue
        keywordstyle=[2]\color{Purple}, %  function arguments purple
        keywordstyle=[3]\color{Blue}\underbar, % Custom functions underlined and blue
        identifierstyle=, % Nothing special about identifiers                                         
        commentstyle=\usefont{T1}{pcr}{m}{sl}\color{MyDarkGreen}\small, % Comments small dark green courier font
        stringstyle=\color{Purple}, % Strings are purple
        showstringspaces=false, % Don't put marks in string spaces
        tabsize=4, % 5 spaces per tab
       	%
        morecomment=[l][\color{Blue}]{...}, % Line continuation (...) like blue comment
        numbers=left, % Line numbers on left
        firstnumber=1, % Line numbers start with line 1
        numberstyle=\tiny\color{Blue}, % Line numbers are blue and small
        stepnumber=2 % Line numbers go in steps of 5
}
\lstset{language=Python, % Use Python in this example
        frame=single, % Single frame around code
        basicstyle=\small\ttfamily, % Use small true type font
        keywordstyle=[1]\color{Blue}\bf, % functions bold and blue
        keywordstyle=[2]\color{Purple}, % function arguments purple
        keywordstyle=[3]\color{Blue}\underbar, % Custom functions underlined and blue
        identifierstyle=, % Nothing special about identifiers                                         
        commentstyle=\usefont{T1}{pcr}{m}{sl}\color{MyDarkGreen}\small, % Comments small dark green courier font
        stringstyle=\color{Purple}, % Strings are purple
        showstringspaces=false, % Don't put marks in string spaces
        tabsize=4, % 5 spaces per tab
       	%
        morecomment=[l][\color{Blue}]{...}, % Line continuation (...) like blue comment
        numbers=left, % Line numbers on left
        firstnumber=1, % Line numbers start with line 1
        numberstyle=\tiny\color{Blue}, % Line numbers are blue and small
        stepnumber=2 % Line numbers go in steps of 2
}

\lstset{language=Perl, % Use Perl in this example
        frame=single, % Single frame around code
        basicstyle=\small\ttfamily, % Use small true type font
        keywordstyle=[1]\color{Blue}\bf, % Perl functions bold and blue
        keywordstyle=[2]\color{Purple}, % Perl function arguments purple
        keywordstyle=[3]\color{Blue}\underbar, % Custom functions underlined and blue
        identifierstyle=, % Nothing special about identifiers                                         
        commentstyle=\usefont{T1}{pcr}{m}{sl}\color{MyDarkGreen}\small, % Comments small dark green courier font
        stringstyle=\color{Purple}, % Strings are purple
        showstringspaces=false, % Don't put marks in string spaces
        tabsize=5, % 5 spaces per tab
        %
        % Put standard Perl functions not included in the default language here
        morekeywords={rand},
        %
        % Put Perl function parameters here
        morekeywords=[2]{on, off, interp},
        %
        % Put user defined functions here
        morekeywords=[3]{test},
       	%
        morecomment=[l][\color{Blue}]{...}, % Line continuation (...) like blue comment
        numbers=left, % Line numbers on left
        firstnumber=1, % Line numbers start with line 1
        numberstyle=\tiny\color{Blue}, % Line numbers are blue and small
        stepnumber=5 % Line numbers go in steps of 5
}

% Creates a new command to include a perl script, the first parameter is the filename of the script (without .pl), the second parameter is the caption
\newcommand{\perlscript}[2]{
\begin{itemize}
\item[]\lstinputlisting[caption=#2,label=#1]{#1.pl}
\end{itemize}
}

\newcommand{\javascript}[2]{
\begin{itemize}
\item[]\lstinputlisting[caption=#2,label=#1]{#1.java}
\end{itemize}
}

\newcommand{\pythonscript}[2]{
\begin{itemize}
\item[]\lstinputlisting[caption=#2,label=#1]{#1.py}
\end{itemize}
}

\newcommand{\cppscript}[2]{
\begin{itemize}
\item[]\lstinputlisting[caption=#2,label=#1]{#1.cpp}
\end{itemize}
}

%----------------------------------------------------------------------------------------
%	DOCUMENT STRUCTURE COMMANDS
%	Skip this unless you know what you're doing
%----------------------------------------------------------------------------------------

% Header and footer for when a page split occurs within a problem environment
\newcommand{\enterProblemHeader}[1]{
\nobreak\extramarks{#1}{#1 continued on next page\ldots}\nobreak
\nobreak\extramarks{#1 (continued)}{#1 continued on next page\ldots}\nobreak
}

% Header and footer for when a page split occurs between problem environments
\newcommand{\exitProblemHeader}[1]{
\nobreak\extramarks{#1 (continued)}{#1 continued on next page\ldots}\nobreak
\nobreak\extramarks{#1}{}\nobreak
}

\setcounter{secnumdepth}{0} % Removes default section numbers
\newcounter{homeworkProblemCounter} % Creates a counter to keep track of the number of problems

\newcommand{\homeworkProblemName}{}
\newenvironment{homeworkProblem}[1][Problem \arabic{homeworkProblemCounter}]{ % Makes a new environment called homeworkProblem which takes 1 argument (custom name) but the default is "Problem #"
\stepcounter{homeworkProblemCounter} % Increase counter for number of problems
\renewcommand{\homeworkProblemName}{#1} % Assign \homeworkProblemName the name of the problem
\section{\homeworkProblemName} % Make a section in the document with the custom problem count
\enterProblemHeader{\homeworkProblemName} % Header and footer within the environment
}{
\exitProblemHeader{\homeworkProblemName} % Header and footer after the environment
}

\newcommand{\problemAnswer}[1]{ % Defines the problem answer command with the content as the only argument
\noindent\framebox[\columnwidth][c]{\begin{minipage}{0.98\columnwidth}#1\end{minipage}} % Makes the box around the problem answer and puts the content inside
}

\newcommand{\homeworkSectionName}{}
\newenvironment{homeworkSection}[1]{ % New environment for sections within homework problems, takes 1 argument - the name of the section
\renewcommand{\homeworkSectionName}{#1} % Assign \homeworkSectionName to the name of the section from the environment argument
\subsection{\homeworkSectionName} % Make a subsection with the custom name of the subsection
\enterProblemHeader{\homeworkProblemName\ [\homeworkSectionName]} % Header and footer within the environment
}{
\enterProblemHeader{\homeworkProblemName} % Header and footer after the environment
}

%----------------------------------------------------------------------------------------
%	NAME AND CLASS SECTION
%----------------------------------------------------------------------------------------

\newcommand{\hmwkTitle}{Java/C++/Python Exercices} % Assignment title
\newcommand{\hmwkDueDate}{Thursday,\ October\ 29,\ 2015} % Due date
\newcommand{\hmwkClass}{Continuous Integration Seminar} % Course/class
\newcommand{\hmwkClassTime}{13h30} % Class/lecture time
\newcommand{\hmwkClassInstructor}{} % Teacher/lecturer
\newcommand{\hmwkAuthorName}{Maurice Bremond, Gaetan Harter, David Parsons -- SED Inria Rhone-Alpes} % Your name

%----------------------------------------------------------------------------------------
%	TITLE PAGE
%----------------------------------------------------------------------------------------

\title{
%\vspace{2in}
\textsc{\hmwkClass}\\
\textbf{\hmwkTitle}\\
%\normalsize\vspace{0.1in}\small{\hmwkDueDate}\\
%\vspace{0.1in}\large{\textit{\hmwkClassInstructor\ \hmwkClassTime}}
%\vspace{3in}
}

\author{\textsf{\hmwkAuthorName}}
\date{\textit{\hmwkDueDate}} % Insert date here if you want it to appear below your name

%----------------------------------------------------------------------------------------

\begin{document}

\maketitle

%----------------------------------------------------------------------------------------
%	TABLE OF CONTENTS
%----------------------------------------------------------------------------------------

%\setcounter{tocdepth}{1} % Uncomment this line if you don't want subsections listed in the ToC

%\newpage
\tableofcontents
%\newpage


%----------------------------------------------------------------------------------------
%	PROBLEM 1
%----------------------------------------------------------------------------------------

% To have just one problem per page, simply put a \clearpage after each problem
\begin{homeworkProblem}[Java Exercice]

\section{Local Test Part}

\subsection{Preamble}

For the Java exercice, we take advantage of Maven environment for building and testing.

So make sure you install \texttt{Maven} tool and that your \texttt{Java} environment is well setup (\texttt{JAVA\_HOME} variable is correctly set, ...).

\subsection{Setup}

\subsubsection{Clone the git repository from inria forge and create your own branch}
\begin{lstlisting}
$ git clone git+ssh://<yourloginhere>@scm.gforge.inria.fr/gitroot/tpcisedra/tpcisedra.git 
$ cd tpcisedra/java
$ git checkout -b <yournamehere>
\end{lstlisting}

\
You should have retrieved the following file tree :\\
{\centering\small
            \begin{minipage}[t]{0.3\linewidth}
\dirtree{%
.1 tpcisedra/java.
.2 pom.xml.
.2 src.
.3 main.
.4 java.
.5 fr.
.6 inria.
.7 sed.
.8 Sphere.java.
.3 tst.
.4 java.
.5 fr.
.6 inria.
.7 sed.
.8 SphereTest.java.
}
\end{minipage}
}


The project is made up of :
\begin{itemize}
\item A Maven ``Project Object Model'' file named \texttt{pom.xml};
\item A file \texttt{Sphere.java} implementing the class \texttt{Sphere};
\item A file \texttt{SphereTest.java} implementing its test class \texttt{SphereTest}.
\end{itemize}

\subsubsection{Check that you can build the project}
\begin{lstlisting}
$ pwd
yourpath/tpcisedra/java
$ mvn package
\end{lstlisting}

You should see these lines (among others):
\begin{lstlisting}
[INFO] ------------------------------------------------------------------------
[INFO] BUILD SUCCESS
[INFO] ------------------------------------------------------------------------
\end{lstlisting}



%--------------------------------------
\subsection{Testing your code}

\subsubsection{Run the (single) test}
\begin{lstlisting}
$ mvn test
\end{lstlisting}

You should see something like
\begin{lstlisting}
-------------------------------------------------------
 T E S T S
-------------------------------------------------------
Running fr.inria.sed.SphereTest
Tests run: 1, Failures: 0, Errors: 0, Skipped: 0, Time elapsed: 0.069 sec

Results :

Tests run: 1, Failures: 0, Errors: 0, Skipped: 0

[INFO] ------------------------------------------------------------------------
[INFO] BUILD SUCCESS
[INFO] ------------------------------------------------------------------------
\end{lstlisting}

\subsubsection{Exercice 1}

Have a look at the code of the method \texttt{Sphere::computeVolume()}.\\
To do so, edit \texttt{tpcisedra/java/src/main/java/fr/inria/sed/Sphere.java}.

The code should be like that :
\begin{lstlisting}
public double computeVolume1() {
  return 4 * Math.PI * Math.pow(radius_, 3) / 3;
}
\end{lstlisting}


We might want the value : \verb?4 * Math.PI / 3? to be computed once and for all.

TODO : 
\begin{itemize}
\item Extract this value into a class data member
\item Run the test again. It should fail.
\end{itemize} 

Here is the output you should get :
\begin{lstlisting}
-------------------------------------------------------
 T E S T S
-------------------------------------------------------
Running fr.inria.sed.SphereTest
Tests run: 1, Failures: 1, Errors: 0, Skipped: 0, Time elapsed: 0.076 sec <<< FAILURE!

Results :

Failed tests:   testComputeVolume(fr.inria.sed.SphereTest):
    expected:<14.137166941154069> but was:<14.137166941154067>

Tests run: 1, Failures: 1, Errors: 0, Skipped: 0

[INFO] ------------------------------------------------------------------------
[INFO] BUILD FAILURE
[INFO] ------------------------------------------------------------------------
\end{lstlisting}

\subsubsection{Exercice 2}

Why does the test in \texttt{SphereTest.java} fail ?

This is because of floating point arithmetics \emph{rouding errors}.

Our response to this issue strongly depends on what kind of application we are working on :
\begin{itemize}
\item In some cases, we want to be aware that something has changed, even when the change is the tiniest. In that case, the test we already have is just what we want.
\item In other cases, we need not worry about such a small difference and hence do not want to be bothered by tests complaining.
\end{itemize}

TODO :
\begin{itemize}
\item Find a way to modify the appropiate test in \texttt{SphereTest.java}
so small rounding differences do not cause errors.
\end{itemize}


%--------------------------------------
\subsection{Test Coverage}

At this stage, all our tests pass. But what does that mean regarding our application ?

Not much you may say, but can you quantify it and will you be able to tell on a real project ?

This is when \textbf{test coverage} becomes handy.

\subsubsection{Run test coverage}

Run the following \texttt{maven} command :
\begin{lstlisting}
$ pwd 
yourpath/tpcisedra/java
$ mvn cobertura:cobertura
\end{lstlisting}

Cobertura should have produced test coverage results in the following directory :\\
\texttt{yourpath/tpcisedra/java/target/site/cobertura}


Use your favorite WEB browser (firefox, chrome, iceweasel, \...) to visualize the generated results of the test coverage
\begin{lstlisting}
$ firefox target/site/cobertura/index.html &
\end{lstlisting}

\subsubsection{Exercice 3}

\begin{itemize}
\item Populate your tests to achieve 100\% test coverage.\\
Hint: you may not need any additional tests ;)
\end{itemize}


\subsubsection{[Optional] Integrate cobertura to your reporting local website}

Add the following to the file  \texttt{pom.xml}
\begin{lstlisting}
  <reporting>
    <plugins>
      <plugin>
        <groupId>org.codehaus.mojo</groupId>
        <artifactId>cobertura-maven-plugin</artifactId>
        <version>2.7</version>
      </plugin>
    </plugins>
  </reporting>
\end{lstlisting}

The report generation is now included in the build lifecycle (``site'' phase).

\subsubsection{Exercice 3bis}

\begin{itemize}
\item To generate the report :
\begin{lstlisting}
$ pwd
yourpath/tpcisedra/java
$ mvn site
\end{lstlisting}

\item Use your favorite WEB browser (firefox, chrome, iceweasel, \...) to visualize the generated report :
\begin{lstlisting}
iceweasel target/site/project-reports.html &
\end{lstlisting}
\end{itemize}

\subsubsection{Add some new class to our project}

Let's add a new class \texttt{Alphabet} and its test class \texttt{AlphabetTest} to our project
\begin{lstlisting}
$ git merge alpha
\end{lstlisting}


\subsubsection{Exercice 4}
\begin{itemize}
\item Generate and visualize the corresponding coverage report
\item Make what changes are necessary to achieve 100\% test coverage
\end{itemize}

%--------------------------------------
\subsection{Stylecheck}

\subsubsection{Run a style checker}

Let's try and run a style checker :
\begin{lstlisting}
$ pwd 
yourpath/tpcisedra/java
$ mvn checkstyle:checkstyle
\end{lstlisting}
We get plenty of errors with the default style \verb?sun_checks.xml?.

Our coding style is close to Google style than it is to Sun style.

\subsubsection{Run Google style checker}

Try running with Google checks (provided by the plugin)
\begin{lstlisting}
$ pwd 
yourpath/tpcisedra/java
$ mvn checkstyle:checkstyle -Dcheckstyle.config.location=google_checks.xml
\end{lstlisting}
NOTE: this can also be achieved by adding the following lines to your pom.xml:
\begin{lstlisting}
  <properties>
    <checkstyle.config.location>checkstyle-test.xml</checkstyle.config.location>
  </properties>
\end{lstlisting}

It's getting better but we might want to make a few changes to this default behaviour.

\subsubsection{Exercice 5}

\begin{itemize}
\item Retrieve a local copy of google checks
\begin{lstlisting}
$ wget https://raw.githubusercontent.com/checkstyle/checkstyle/
     18f6ebbcad23e88e3ae30fc79be464b8b7772a0d/google_checks.xml
\end{lstlisting}
\item  Modify the file \texttt{google\_checks.xml} so that it accepts single character
parameter names.
\item You may also consider removing trailing underscores from data members or amending checkstyle.xml
\end{itemize}

\subsubsection{[Optional] Integrate checkstyle to your reporting local website}

Add the following to the file \texttt{pom.xml}
\begin{lstlisting}
  <reporting>
    <plugins>
      <plugin>
        <groupId>org.apache.maven.plugins</groupId>
        <artifactId>maven-checkstyle-plugin</artifactId>
        <version>2.16</version>
      </plugin>
    </plugins>
  </reporting>
\end{lstlisting}

The check style report is now included in the build lifecycle (``site'' phase).

\subsubsection{Exercice 5bis}

\begin {itemize}
\item To generate the report :
\begin{lstlisting}
$ pwd
yourpath/tpcisedra/java
$ mvn site
\end{lstlisting}
\item Use your favorite WEB browser (firefox, chrome, iceweasel, \...) to visualize the generated report :
\begin{lstlisting}
$ firefox target/site/project-reports.html &
\end{lstlisting}
\end{itemize}

\section{Continuous Integration Part}
\end{homeworkProblem}

\clearpage




\begin{homeworkProblem}[Python Exercice]

\section{Local Test Part}
\section{Continuous Integration Part}
\subsection{Validate on CI}

At this point you have a your python project packaged with setup.py and tests
running managed by tox. This means that running your tests only need to run:

\texttt{tox}

All the dependencies will be installed automatically.

\subsection{Running your first test on CI}

\begin{itemize}
\item Create your job with \texttt{New Item} button on the left menu.
\item For Python create a \texttt{Freestyle project} and put a name without spaces, it always
brings errors in scripts to have path with spaces.
\item Then go to \texttt{Configure}.
\end{itemize}

\subsubsection{Git configuration}

\begin{itemize}
\item Go to \texttt{Source Code Management} and choose \texttt{Git}
\item Put the anonymous git url of the repository \texttt{ADD\_URL\_HERE} and select your branch name \texttt{*/branch\_name}.
\end{itemize}

\subsubsection{Build Step}

\begin{itemize}
\item Add build step : \texttt{Execute Shell} and execute :
\begin{lstlisting}
cd python/
tox
\end{lstlisting}
\item Save the configuration.
\end{itemize}

\subsubsection{Running test}

What is the result ?

See \texttt{Console Output} to see the script execution.

\subsection{Fixing the errors the Python way}

As seen, \texttt{tox} is not installed on the build machine. It can either be installed
manually or as a step of the integration process. The last solution is more in
the process of 'integration' but sometimes you don't have the choice.

\begin{itemize}
\item Remove the build step and replace it with \texttt{Virtualenv Builder}.
This puts the code execution in a separated python environment for the build.
\item Now as first step, install your `tox` dependency:
\begin{lstlisting}
pip install --quiet --upgrade tox
cd python
tox
\end{lstlisting}
\item Save and build, it should be a successful build.
\end{itemize}

See the build output, you can see tests/coverage/pylint/pep8 text results.


\subsection{Adding Build feedback}

A big value of Jenkins is the ability to nicely present your tests results.

The tools used to run tests and code quality where selected because they have output files compatible with Jenkins plugins.

\begin{description}
\item[Note] 
This selection may look restrictive, but it also means that the tools are indeed used by many developers and that it is
not only a one man written test script.
\end{description}

\begin{itemize}
\item \texttt{nosetests} outputs as a JUnit compatible XML: \texttt{nosetests.xml}
\item \texttt{nose-xcover} outputs coverage compatible with cobertura: \texttt{coverage.xml}
\item \texttt{pylint} and \texttt{pep8} output is recognized by 'warnings' and 'violations' plugins.
\end{itemize}

These output parsing is managed in \texttt{Add post-build action}.
%% Maybe print the real plugins names here

\subsection{Add tests results feedback}

\begin{description}
\item[Step \texttt{Publish JUnit test result report}]
\begin{itemize}
\item Test report xml: `**/*tests.xml` to scan for all files ending with 'tests.xml'
\end{itemize}
\end{description}

\subsubsection{Add coverage feedback}

\begin{description}
\item[Step \texttt{Publish Cobertura Coverage Report}]
  \begin{itemize}
  \item Cobertura xml report pattern:  \texttt{**/coverage.xml} to scan for all  \texttt{coverage.xml} files.
  \end{itemize}
\end{description}

As source files are not based at root directory Cobertura fails to locate source
files. 

A solution is to create a symlink to the source directory :
\begin{itemize}
  \item Add a build step \texttt{Execute shell} with:
\begin{lstlisting}
# Hack to help cobertura find source files
ln  -nfs Src/python/abc\_and\_sphere  abc\_and\_sphere
\end{lstlisting}
\end{itemize}

\subsubsection{Add code quality output}

\begin{description}
\item[Step \texttt{Scan for compiler warnings}]
  \begin{itemize}
  \item Scan console log add two parsers, one for \texttt{pylint} and another for \texttt{pep8}.
  \end{itemize}
\end{description}


\subsubsection{Code review by ChuckNorris}

\begin{description}
\item[Step \texttt{Activate Chuck Norris}]
It displays you Chuck Norris facts and a picture of Chuck adapted to your build result (seems like the picture is not displayed ?)
\end{description}


Do not forget to save configuration.


\subsection{Analyze build output}

\begin{itemize}
\item Build the project two times to get a nice displayed output. Plugins need
multiple builds to create graphs.
\end{itemize}


\subsection{Coverage Report}

\begin{itemize}
\item Click on Coverage Report, you can then see per file coverage output.
\end{itemize}


\subsection{Pylint/PEP8}

\begin{itemize}
\item The per file output didn't work ? Why ?
\end{itemize}



\subsection{Further steps}

%% TODO

Automatically run the build:
\begin{itemize}
\item  Never: you only run it manually (not recommended)
\item Periodically: Every night at 3AM may be enough if your code moves slowly
\item At each commit: with repository polling/ with git hook (see how...)
\end{itemize}

Send mails on build failure:
\begin{itemize}
\item  Howto do that ?
\end{itemize}


\subsection{Improvements}

\begin{itemize}
\item Try fixing all your code to get perfect results outputs;*
\item Use tox to run tests on python3 also and make your code compatible with.
\end{itemize}

%%\lipsum[1]

%%Listing \ref{Sphere} shows a Java program.
%%\javascript{Sphere}{Sample Java program with Highlighting}

%%Another program example for Python
%%Listing \ref{Python_example} shows a Python program.
%%\pythonscript{Python_example}{Sample Python program with highlighting}


\end{homeworkProblem}

%----------------------------------------------------------------------------------------
%	PROBLEM 2
%----------------------------------------------------------------------------------------

\begin{homeworkProblem}[C++ Exercice]


%%Extract from C++ program
%%\begin{lstlisting}[name=Function Test, frame=trBL]
%%#include <iostream>
%%#include <basetsd.h>
%%#include <iomanip>
%%#include <cstdlib>
%%using namespace std;
%% 
%%union FloatNum //Here the tag name (FloatNum) is redundant.
%% {
%%   float fx;//4 bytes variable
%%   long  lx;//4 bytes variable
%% }fn;
%% 
%%union DoubleNum
%% {
%%   double dx;  //8 bytes variable
%%   LONG64 lx;  //8 bytes variable
%% }dn;
%% 
%%union LongDoubleNum
%% {
%%   long double dx;  //12 bytes variable
%%   long  lx[3]; // 3 * 4 bytes variable
%% }ldn;
%% 
%%int main()
%%{
%%    fn.fx = -118.6253433; //variable assignment declaration statement
%%   //show size of float
%%    cout << "\nsize of float = " << dec << sizeof(fn.fx) << endl;
%%    cout << setprecision(10) << fn.fx << " = 0x" << hex << fn.lx << endl;
%%
%%    dn.dx =  112.6255678;  //assign value to a variable
%%    //show size of double
%%    cout << "\nsize of double = " << dec << sizeof(dn.dx) << endl;
%%    cout << dn.dx <<"  = 0x" << hex << dn.lx << endl;
%%
%%    ldn.dx = -12.61256125;  //assign value to a variable
%%    //show size of long double
%%    cout << "\nsize of long double = " << dec << sizeof(ldn.dx) << endl;
%%    cout << setprecision(10) << ldn.dx << " = 0x" 
%%         <<  hex << ldn.lx[2] << ldn.lx[1] << ldn.lx[0] << endl;
%%    return 0;
%%}
%%\end{lstlisting}


%\problemAnswer{

%Answer
%\begin{center}
%\includegraphics[width=0.75\columnwidth]{example_figure} % Example image
%\end{center}

%\lipsum[3-5]
%}
\end{homeworkProblem}

%----------------------------------------------------------------------------------------

\end{document}
